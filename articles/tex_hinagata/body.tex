\subtitle{ヘッダの見出し}
\author{ほげ}

\chapter{記事の書き方(\TeX 編)}

\section{ファイル構成}

この雛形に含まれているファイルは次のような意味になっています。

\begin{description}
  \item[\ovalbox{main.tex}]
    クラスファイルやパッケージのロードを行うファイル。
  \item[\ovalbox{body.tex}]
    記事の本体の\TeX ファイル。
    このファイルはリポジトリのルートにある\ovalbox{main.tex}からも読み込まれる。
\end{description}

\section{記事を書く}

この\ovalbox{body.tex}を編集すると記事になります。
記事を書いたら、 \lstinline|make| コマンドでビルドできます。

\begin{lstlisting}
make
\end{lstlisting}

これで\ovalbox{main.pdf} が生成されれば成功です。
あとは\ovalbox{main.tex} を編集すれば記事が出来ます。

\section{記事の追加}

作った記事をリポジトリのルートにある\ovalbox{main.tex}に追加する必要がある。
次のような\TeX プログラムを追加する。

\begin{lstlisting}
\setcounter{section}{0}
\makeatletter
\def\input@path{{./articles/<ARTICLE-DIRECTORY-NAME>/}}
\renewcommand\includegraphics[2][]{%
  \latexincludegraphics[#1]{./articles/<ARTICLE-DIRECTORY-NAME>/#2}
}
\renewcommand\bibliography[1]{%
  \latexbibliography{./articles/<ARTICLE-DIRECTORY-NAME>/#1}
}
\makeatother

\documentclass{word-lua}
\usepackage{fontspec}
\usepackage{letltxmacro}
\usepackage[%
  unicode=true,
  breaklinks=true
  bookmarks=true,
  colorlinks=true]{hyperref}
\usepackage{graphics}
\usepackage{listings}
\lstset{
  basicstyle=\ttfamily,
  commentstyle=\textit,
  frame=trBL,
  numbers=left,
  breaklines=true,                 % sets automatic line breaking
  language=TeX,                 % the language of the code
  title=\lstname                   % show the filename of files included with \lstinputlisting; also try caption instead of title
}
\usepackage{fancybox}
\usepackage{url}

\LetLtxMacro\latexincludegraphics\includegraphics
\LetLtxMacro\latexbibliography\bibliography

\providecommand{\tightlist}{%
  \setlength{\itemsep}{0pt}\setlength{\parskip}{0pt}}

\begin{document}

\makeatletter
\def\input@path{{./articles/tex_hinagata/}}
\renewcommand\includegraphics[2][]{%
  \latexincludegraphics[#1]{./articles/tex_hinagata/#2}
}
\renewcommand\bibliography[1]{%
  \latexbibliography{./articles/tex_hinagata/#1}
}
\makeatother

\subtitle{ヘッダの見出し}
\author{ほげ}

\chapter{記事の書き方(\TeX 編)}

\section{ファイル構成}

この雛形に含まれているファイルは次のような意味になっています。

\begin{description}
  \item[\ovalbox{main.tex}]
    クラスファイルやパッケージのロードを行うファイル。
  \item[\ovalbox{body.tex}]
    記事の本体の\TeX ファイル。
    このファイルはリポジトリのルートにある\ovalbox{main.tex}からも読み込まれる。
\end{description}

\section{記事を書く}

この\ovalbox{body.tex}を編集すると記事になります。
記事を書いたら、 \lstinline|make| コマンドでビルドできます。

\begin{lstlisting}
make
\end{lstlisting}

これで\ovalbox{main.pdf} が生成されれば成功です。
あとは\ovalbox{main.tex} を編集すれば記事が出来ます。

\section{記事の追加}

作った記事をリポジトリのルートにある\ovalbox{main.tex}に追加する必要がある。
次のような\TeX プログラムを追加する。

\begin{lstlisting}
\setcounter{section}{0}
\makeatletter
\def\input@path{{./articles/<ARTICLE-DIRECTORY-NAME>/}}
\renewcommand\includegraphics[2][]{%
  \latexincludegraphics[#1]{./articles/<ARTICLE-DIRECTORY-NAME>/#2}
}
\renewcommand\bibliography[1]{%
  \latexbibliography{./articles/<ARTICLE-DIRECTORY-NAME>/#1}
}
\makeatother

\documentclass{word-lua}
\usepackage{fontspec}
\usepackage{letltxmacro}
\usepackage[%
  unicode=true,
  breaklinks=true
  bookmarks=true,
  colorlinks=true]{hyperref}
\usepackage{graphics}
\usepackage{listings}
\lstset{
  basicstyle=\ttfamily,
  commentstyle=\textit,
  frame=trBL,
  numbers=left,
  breaklines=true,                 % sets automatic line breaking
  language=TeX,                 % the language of the code
  title=\lstname                   % show the filename of files included with \lstinputlisting; also try caption instead of title
}
\usepackage{fancybox}
\usepackage{url}

\LetLtxMacro\latexincludegraphics\includegraphics
\LetLtxMacro\latexbibliography\bibliography

\providecommand{\tightlist}{%
  \setlength{\itemsep}{0pt}\setlength{\parskip}{0pt}}

\begin{document}

\makeatletter
\def\input@path{{./articles/tex_hinagata/}}
\renewcommand\includegraphics[2][]{%
  \latexincludegraphics[#1]{./articles/tex_hinagata/#2}
}
\renewcommand\bibliography[1]{%
  \latexbibliography{./articles/tex_hinagata/#1}
}
\makeatother

\input{articles/tex_hinagata/body.tex}

\makeatletter
\def\input@path{{./articles/markdown_hinagata/}}
\renewcommand\includegraphics[2][]{%
  \latexincludegraphics[#1]{./articles/markdown_hinagata/#2}
}
\renewcommand\bibliography[1]{%
  \latexbibliography{./articles/markdown_hinagata/#1}
}
\makeatother

\input{articles/markdown_hinagata/main.tex}

\end{document}

\end{lstlisting}

この時、もし\lstinline|\usepackage|が増えた場合は次の二つを実行してください。

\begin{itemize}
  \item リポジトリのルートにある\ovalbox{main.tex}にも\lstinline|\usepackage|を追加する
  \item \ovalbox{.travis.yml}の\lstinline|sudo tlmgr install|に必要があればパッケージを追加する
\end{itemize}


\makeatletter
\def\input@path{{./articles/markdown_hinagata/}}
\renewcommand\includegraphics[2][]{%
  \latexincludegraphics[#1]{./articles/markdown_hinagata/#2}
}
\renewcommand\bibliography[1]{%
  \latexbibliography{./articles/markdown_hinagata/#1}
}
\makeatother

\subtitle{ヘッダの見出し}
\author{ほげ}

\subtitle{ヘッダの見出し}
\author{ほげ}

\chapter{記事の書き方(\TeX 編)}

\section{ファイル構成}

この雛形に含まれているファイルは次のような意味になっています。

\begin{description}
  \item[\ovalbox{main.tex}]
    クラスファイルやパッケージのロードを行うファイル。
  \item[\ovalbox{body.tex}]
    記事の本体の\TeX ファイル。
    このファイルはリポジトリのルートにある\ovalbox{main.tex}からも読み込まれる。
\end{description}

\section{記事を書く}

この\ovalbox{body.tex}を編集すると記事になります。
記事を書いたら、 \lstinline|make| コマンドでビルドできます。

\begin{lstlisting}
make
\end{lstlisting}

これで\ovalbox{main.pdf} が生成されれば成功です。
あとは\ovalbox{main.tex} を編集すれば記事が出来ます。

\section{記事の追加}

作った記事をリポジトリのルートにある\ovalbox{main.tex}に追加する必要がある。
次のような\TeX プログラムを追加する。

\begin{lstlisting}
\setcounter{section}{0}
\makeatletter
\def\input@path{{./articles/<ARTICLE-DIRECTORY-NAME>/}}
\renewcommand\includegraphics[2][]{%
  \latexincludegraphics[#1]{./articles/<ARTICLE-DIRECTORY-NAME>/#2}
}
\renewcommand\bibliography[1]{%
  \latexbibliography{./articles/<ARTICLE-DIRECTORY-NAME>/#1}
}
\makeatother

\input{articles/<ARTICLE-DIRECTORY-NAME>/main.tex}
\end{lstlisting}

この時、もし\lstinline|\usepackage|が増えた場合は次の二つを実行してください。

\begin{itemize}
  \item リポジトリのルートにある\ovalbox{main.tex}にも\lstinline|\usepackage|を追加する
  \item \ovalbox{.travis.yml}の\lstinline|sudo tlmgr install|に必要があればパッケージを追加する
\end{itemize}




\end{document}

\end{lstlisting}

この時、もし\lstinline|\usepackage|が増えた場合は次の二つを実行してください。

\begin{itemize}
  \item リポジトリのルートにある\ovalbox{main.tex}にも\lstinline|\usepackage|を追加する
  \item \ovalbox{.travis.yml}の\lstinline|sudo tlmgr install|に必要があればパッケージを追加する
\end{itemize}

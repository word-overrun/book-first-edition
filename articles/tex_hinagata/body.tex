\subtitle{ヘッダの見出し}
\author{ほげ}

\chapter{記事の書き方}

\section{まずはじめに}

\subsection{デフォルトオプション: p\LaTeX を使う}

$article\_name$は適当な名前として、以下のようなコマンドでブランチを分けましょう。

\begin{lstlisting}[mathescape]
git submodule update --init
cd ./articles
cp -r ./tex_hinagata ./my-article-name
cd ./my-article-name
autoconf
./configure
\end{lstlisting}

\subsection{選択: Lua\LaTeX を使う}

WORD では新たに Lua\LaTeX が使えるようになりました。
使い方は、\lstinline|./configure|のかわりに\lstinline|./configure --enable-luatex| としてください。

\section{記事を書く}

記事を書いたら、 \lstinline|make| コマンドでビルドできます。

\begin{lstlisting}
git add *
make
\end{lstlisting}

これで\ovalbox{main.pdf} が生成されれば成功です。
あとは\ovalbox{main.tex} を編集すれば記事が出来ます。

\section{Git サーバに push する}

記事のキリの良いところで\lstinline|git push|するといいのですが、最初の push の時には、
origin\footnote{ここでは git サーバである dev.word-ac.net のことです}
に新しいブランチを登録する必要があります。それは以下のようにしましょう。

\begin{lstlisting}[mathescape]
git push origin personal/$username$/$article\_name$
\end{lstlisting}

push を成功させた場合には、ビルドの結果が slack\footnote{\url{https://word-ac.slack.com}} の \#jenkins チャンネルに流れます。
slack を見ていない場合は、\url{http://dev.word-ac.net/jenkins/job/LaTeX/} および \url{http://dev.word-ac.net/gitweb/} を見ると良いでしょう。

\section{トラブルシューティング}

\subsection{偶数頁}

編集作業をしていると、レイアウトの問題で偶数頁から開始していただくことがあります。その場合の対処法は、 TeX の処理系によって以下のように異なります。

\subsubsection{p\LaTeX を使う場合}

その場合は、プレアンブルに以下を追加してください。

\begin{lstlisting}[mathescape]
\setcounter{page}{2}
\end{lstlisting}

\subsubsection{Lua\LaTeX を使う場合}

\lstinline|\documentclass| のオプションに \ovalbox{swapheader} をつけることで簡単にできます。

\begin{lstlisting}[mathescape]
\documentclass[swapheader]{word-lua}
\end{lstlisting}

\section{鍵の登録}

Git サーバに鍵を登録しないと、 push できません。もしそれが原因でつまっている場合には、誰か権限を持っていそうな人に頼んで登録してもらいましょう。2016年6月現在では、
pi8027, yyu, ioriveur, shinkbr, osyoyu, chris, nymphium が部員を登録できます。鍵が変わった場合も声をかけましょう。

\section{他の問題について}

問題があれば slack の \#latex チャンネルや、編集会議で聞くと良いでしょう。

直接詳しい人に SNS で聞く場合、 @\_yyu\_\footnote{\url{https://twitter.com/_yyu_}}へ投げると早い。
word-lua に関しては@Nymphium\footnote{\url{https://twitter.com/Nymphium}}か@azuma962\footnote{\url{https://twitter.com/azuma962}}へ。

